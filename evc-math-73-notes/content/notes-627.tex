\chapter{June 27, 2024}

\section{Multivariable Functions Continued}

Now, we proceed to define continuity for multivariable functions. For the sake of simplicity, we offer the definition of a function with two input variables (the most common type of function we will see), but the results are easily generalizable to further dimensions.

\begin{definition}
    A function $f(x, y)$ is continuous at $(a, b)$ if and only if
    \[\lim_{(x, y) \to (a, b)} f(x, y) = f(a, b).\]
\end{definition}

We also have the notion of continuity in a region.

\begin{definition}
    A function $f(x, y)$ is continuous on $D$ if and only if
    \[\lim_{(x, y) \to (a, b)} f(x, y) = f(a, b) \sforall (a, b) \in D.\]
\end{definition}

Now, we introduce the topic of partial derivatives, which is essentially taking the derivative with respect to one input variable in a multiavariable function, treating the others as constants.

\begin{example}
    Given a multivariable function $f(x, y) = xy + x^{2} - y^{2} + \cos x$, find its partial derivative with respect to $x$.

    \begin{soln}
        We treat $y$ as a constant and proceed to differentiate as if it was a single-input function.
        \begin{align*}
            \frac{\partial f}{\partial x} &= \frac{\partial}{\partial x}\left(xy + x^{2} - y^{2} + \cos x\right) \\
            &= y + 2x - \sin x.
        \end{align*}
    \end{soln}
\end{example}

\begin{remark}
    The notation $\frac{\partial f}{\partial x}$ is equivalent to the notation $f_{x}(x, y)$, which may also be used.
\end{remark}

\begin{proposition}
    A function $f(x, y)$ is smooth at $(a, b)$ if and only if
    \[f_{xy}(a, b) = f_{yx}(a, b).\]
\end{proposition}