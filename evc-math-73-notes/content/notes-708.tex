\chapter{July 8, 2024}

\section{Lagrange Multipliers}

If we have a function $f(x_{1}, x_{2}, \dots, x_{n})$ that we want to optimize under the constraint $g(x_{1}, x_{2}, \dots, x_{n}) = c$, we can do so using the Lagrangian function
\[\mathcal{L}(x_{1}, x_{2}, \dots, x_{n}, \lambda) = f(x_{1}, x_{2}, \dots, x_{n}) - \lambda[g(x_{1}, x_{2}, \dots, x_{n}) - c].\]
By equating the partial derivative with respect to each input variable to zero and solving the resulting system of equations, we can find input values yielding critical outputs.

\begin{remark}
    In other words, by taking the partial derivatives we set
    \[\nabla f(x_{1}, x_{2}, \dots, x_{n}) = \lambda\nabla g(x_{1}, x_{2}, \dots, x_{n}).\]
    This essentially means we want the gradient of $f$ to be parallel to the gradient of $g$.
\end{remark}

\begin{example}
    Given the constraint
    \[x + y = 1,\]
    find the extrema of
    \[z = 1 - x^{2} - y^{2}\]
    \begin{soln}
        We have
        \[\mathcal{L}(x, y, \lambda) = 1 - x^{2} - y^{2} + \lambda(x + y - 1),\]
        which means
        \[\frac{\partial \mathcal{L}}{\partial x} = \lambda - 2x\]
        \[\frac{\partial \mathcal{L}}{\partial y} = \lambda - 2y.\]
        By equating to 0,
        \[\lambda = 2x = 2y \implies x = y = \frac{1}{2}.\]
        So, $z = \frac{1}{2}$ as well.
    \end{soln}
\end{example}