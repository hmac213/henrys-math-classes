\chapter{I was Probably Watching Reels when We Learned This}

\section{Line Integrals}

Now we go over line integrals in general, of which there are two types: line integrals of scalar fields and line integrals of vector fields. Let us start with scalar fields.

\begin{definition}
    The line integral of $f(x, y)$ along $C$ is defined by
    \[\oint_{C}f(x, y)ds\]
    where $ds$ is the arclength parameterization defined by
    \[ds = \sqrt{\left(\frac{dx}{dt}\right)^{2} + \left(\frac{dy}{dt}\right)^{2}}dt.\]
\end{definition}

For vector fields, we have the following.

\begin{definition}
    The line integral of $\vec{F}(x, y, z)$ along $C$ is
    \[\oint_{C}\vec{F} \cdot d\vec{r} = \int_{a}^{b}\vec{F}(\vec{r}(t))\ddvec{r}(t)dt\]
    where $t$ ranges from $a$ to $b$.
\end{definition}

Say $\vec{F}(x, y, z) = \left<P(x, y, z), Q(x, y, z), R(x, y, z)\right>$. The line integral above can also be expressed in another common way.

\begin{corollary}
    The line integral of $\vec{F}(x, y, z)$ along $C$ is
    \[\oint_{C}\vec{F} \cdot d\vec{r} = \oint_{C}\left(Pdx + Qdy + Rdz\right)\]
\end{corollary}

The above is shorthand for the sum of the respective components of the input of $\vec{F}$.

\section{Green's Theorem}

Here, I offer the statement and an elementary proof of Green's Theorem.

\begin{theorem}[Green's Theorem]
    For a region $D$ bounded by a piecewise smooth and positively oriented curve $C$ in a plane, if $M$ and $N$ are functions of $(x, y)$ defined on an open region containing $D$ and both have continuous first order partial derivatives, then
    \[\oint_{C}(Mdx + Ndy) = \iint\limits_{D}\left(\frac{\partial N}{\partial x} - \frac{\partial M}{\partial y}\right)dxdy.\]
\end{theorem}

\begin{proof}
    We will first prove the validity of Green's theorem on a rectangular region, which we say has arbitrary vertices at $(a, c)$, $(a, d)$, $(b, c)$, and $(b, d)$ where $a < b$ and $c < d$. Starting in the bottom left corner and proceeding counterclockwise, we label the four paths of the rectangle $\ell_{1}$, $\ell_{2}$, $\ell_{3}$, and $\ell_{4}$. First looking at the left side of the theorem statement, we have
    \[\oint_{C}(Mdx + Ndy) = \sum_{k = 1}^{4}\int_{\ell_{k}}(Mdx + Ndy).\]
    Since, $\ell_{1}$ and $\ell_{3}$ are horizontal and $\ell_{2}$ and $\ell_{4}$ are vertical, we note that $dy$ and $dx$ are zero, respectively, in each case. A simplification and an insertion of integration bounds yields
    \begin{align*}
        \oint_{C}(Mdx + Ndy) &= \int_{a}^{b}M(x, c)dx + \int_{c}^{d}N(b, y)dy + \int_{b}^{a}M(x, d)dx + \int_{d}^{c}N(a, y)dy \\
        &= \int_{a}^{b}M(x, c) - M(x, d)dx - \int_{c}^{d}N(a, y) - N(b, y)dy.
    \end{align*}
    Now, we look at the right side. By direct calculation,
    \begin{align*}
        \iint\limits_{D}\left(\frac{\partial N}{\partial x} - \frac{\partial M}{\partial y}\right)dxdy &= \int_{c}^{d}\int_{a}^{b}\left(\frac{\partial N}{\partial x} - \frac{\partial M}{\partial y}\right)dxdy \\
        &= \int_{c}^{d}\int_{a}^{b}\frac{\partial N}{\partial x}dxdy - \int_{a}^{b}\int_{c}^{d}\frac{\partial M}{\partial y}dydx \\
        &= \int_{c}^{d} N(x, y) \biggr\rvert_{x = a}^{x = b} dy - \int_{a}^{b} M(x, y) \biggr\rvert_{y = c}^{y = d} dx \\
        &= \int_{c}^{d} N(b, y) - N(a, y) dy - \int_{a}^{b} M(x, d) - M(x, c) dx.
    \end{align*}
    As we have succesfully shown the left and right-hand sides to be equal, we have proven Green's theorem for a rectangular region. When we place two of these regions side-by-side, the integral over the shared edge evaluates to zero because the paths are antiparallel and perfectly overlapping.
    So, only edges on the border (ie. not touching another rectangle) are evaluated to non-zero quantities. Since any surface can be approximated with an infinitely packed collection of rectangles, we see that the case of a rectangle easily generalizes to any piecewise smooth plane curve.
\end{proof}